\documentclass[10pt,
  aspectratio=169,
  serif,
  mathserif,
  professionalfont,
  compress,
  handout,
  % table,
  % svgnames
  ]{beamer}\usepackage[]{graphicx}\usepackage[]{color}
% maxwidth is the original width if it is less than linewidth
% otherwise use linewidth (to make sure the graphics do not exceed the margin)
\makeatletter
\def\maxwidth{ %
  \ifdim\Gin@nat@width>\linewidth
    \linewidth
  \else
    \Gin@nat@width
  \fi
}
\makeatother

\definecolor{fgcolor}{rgb}{1, 1, 0.941}
\newcommand{\hlnum}[1]{\textcolor[rgb]{0.804,0.718,0.71}{#1}}%
\newcommand{\hlstr}[1]{\textcolor[rgb]{0.604,0.753,0.804}{#1}}%
\newcommand{\hlcom}[1]{\textcolor[rgb]{0.439,0.502,0.565}{#1}}%
\newcommand{\hlopt}[1]{\textcolor[rgb]{1,1,0.941}{#1}}%
\newcommand{\hlstd}[1]{\textcolor[rgb]{1,1,0.941}{#1}}%
\newcommand{\hlkwa}[1]{\textcolor[rgb]{0.941,0.902,0.549}{#1}}%
\newcommand{\hlkwb}[1]{\textcolor[rgb]{1,0.871,0.678}{#1}}%
\newcommand{\hlkwc}[1]{\textcolor[rgb]{0.545,0.941,0.702}{#1}}%
\newcommand{\hlkwd}[1]{\textcolor[rgb]{0.545,0.941,0.902}{#1}}%
\let\hlipl\hlkwb

\usepackage{framed}
\makeatletter
\newenvironment{kframe}{%
 \def\at@end@of@kframe{}%
 \ifinner\ifhmode%
  \def\at@end@of@kframe{\end{minipage}}%
  \begin{minipage}{\columnwidth}%
 \fi\fi%
 \def\FrameCommand##1{\hskip\@totalleftmargin \hskip-\fboxsep
 \colorbox{shadecolor}{##1}\hskip-\fboxsep
     % There is no \\@totalrightmargin, so:
     \hskip-\linewidth \hskip-\@totalleftmargin \hskip\columnwidth}%
 \MakeFramed {\advance\hsize-\width
   \@totalleftmargin\z@ \linewidth\hsize
   \@setminipage}}%
 {\par\unskip\endMakeFramed%
 \at@end@of@kframe}
\makeatother

\definecolor{shadecolor}{rgb}{.97, .97, .97}
\definecolor{messagecolor}{rgb}{0, 0, 0}
\definecolor{warningcolor}{rgb}{1, 0, 1}
\definecolor{errorcolor}{rgb}{1, 0, 0}
\newenvironment{knitrout}{}{} % an empty environment to be redefined in TeX

\usepackage{alltt}

% Tamanho de fonte e distância entre linhas.
\renewenvironment{knitrout}{
  \renewcommand{\baselinestretch}{0.75}%\tiny
}{}

%-----------------------------------------------------------------------
% Pacotes padrões.

% Fontes.
\usepackage{palatino}
\usepackage{eulervm}
\usepackage{inconsolata}

% Esses pacotes dão clash.
% http://tex.stackexchange.com/questions/51488/option-clash-with-xcolor-and-tikz
% \usepackage{xcolor} %% opções no \documentclass{} para evitar clash.
% \definecolor{mycolor}{rgb}{0.13,0.53,0.53}
% \definecolor{mycolor2}{rgb}{0.725,0,0.18}

\usepackage{hyperref}
 \hypersetup{colorlinks, allcolors=., urlcolor=structure}
% \hypersetup{colorlinks}

\usepackage[brazil]{babel}
\usepackage[utf8]{inputenc}
\usepackage{graphicx}
\usepackage{amsmath, amsfonts, amssymb, amsxtra, amsthm, icomma}
\usepackage{geometry, calc, setspace, indentfirst}
% \usepackage{colortbl}
\usepackage{enumerate}
\usepackage{float}

\usepackage{subfigure}

\usepackage[hang]{caption}
\captionsetup{font=footnotesize,
  labelfont=footnotesize,
  labelsep=period}

% Listas em duas colulas.
\usepackage{multicol}
\newenvironment{itemize2}{%
  \vspace*{-1em}
  \begin{itemize}
    \begin{multicols}{2}
    }{%
    \end{multicols}
  \end{itemize}
}

% Texto no corpo do beamer justificado.
\usepackage{ragged2e}
\justifying

%-----------------------------------------------------------------------

% A lot of options:
% http://latex-community.org/forum/viewtopic.php?f=55&t=17646
\useoutertheme[
  width=60pt,
  height=30pt,
  right,
  hideothersubsections
  ]{sidebar}

\makeatletter
\setbeamertemplate{caption}[numbered]
\setbeamertemplate{section in toc}[sections numbered]
\setbeamertemplate{subsection in toc}[subsections numbered]
\setbeamertemplate{sections/subsections in toc}[ball]{}
\setbeamertemplate{section in sidebar right}[sections numbered]
%\setbeamertemplate{frametitle continuation}{\gdef\beamer@frametitle{}}
\setbeamertemplate{navigation symbols}{} %% Retira a barra de navegação.
\setbeamertemplate{footline}[frame number]
% \setbeamertemplate{blocks}[rounded][shadow=FALSE]
% \setbeamercolor{block title}{fg=structure, bg=mycolor!20!white}
\makeatother

% Frames com sessão e/ou subsessão.
% \addtobeamertemplate{frametitle}{
%   \let\insertframetitle\insertsubsectionhead}{}
% \makeatletter
% \CheckCommand*\beamer@checkframetitle{
%   \@ifnextchar\bgroup\beamer@inlineframetitle{}}
% \renewcommand*\beamer@checkframetitle{
%   \global\let\beamer@frametitle\relax\@ifnextchar\bgroup
%   \beamer@inlineframetitle{}}
% \makeatother

\setbeamertemplate{frametitle}
{
    \nointerlineskip
    \begin{beamercolorbox}[sep=0.3cm,ht=1.8em,wd=\paperwidth]{frametitle}
        \vbox{}\vskip-2ex%
        \strut\insertframetitle\strut
        \vskip-0.8ex%
    \end{beamercolorbox}
}


%-----------------------------------------------------------------------
% Comandos.

\newcommand{\n}[1]{\textbf{#1}}

%-----------------------------------------------------------------------

\AtBeginSection[]{
  \begin{frame}[c,allowframebreaks]
    \begin{center}
      {\thesection} \\ \vspace{0.3cm}
      \parbox{0.6\textwidth}{
        \centering {\Large \textcolor{structure}{\insertsection}}}\\
    \end{center}
  \end{frame}
}

%-----------------------------------------------------------------------
% Definições dos proprietários.

\title[]{
  \Large Testes de comparações múltiplas em modelos multivariados de covariância linear generalizada}

\subtitle{\vspace{0.5cm}Tópicos avançados em modelagem estatística}

\author[Lineu Alberto]{\small
  Lineu Alberto Cavazani de Freitas
}

 \institute[UFPR]{
 Trabalho final\\
%    PPG Informática \\
%   Data Science \& Big Data \\

\vspace{1em}
   Universidade Federal do Paraná
% 
%   \vspace{1em}
%   \href{}{https://lineu96.github.io/st/}\\
%   \texttt{lineuacf@gmail.com}
 }
 \date{}

\logo{\includegraphics[width=1.5cm]{img/dsbd1x4-rect.png}} 

\usebackgroundtemplate{
  \includegraphics[width=\paperwidth]{img/ufpr-fundo.jpg}
}

\titlegraphic{
  \vspace{-3.4em}
%  \includegraphics[height=1.8cm]{img/capes_tp2.png}\hspace{2em}
  %\includegraphics[height=1.8cm]{img/ufpr-transparent.png}\hspace{2em}
%  \includegraphics[height=1.8cm]{img/dsbd-2x2-trans.png}
}

%---- preamble-refs.tex ------------------------------------------------

% Bibliography.

% %\usepackage[style=authoryear]{biblatex}
% \usepackage[authordate, bibencoding=auto, strict, backend=biber, natbib]{biblatex-chicago}
%
% % Use:
% %   \cite{<ref>}
% %   \parencite{<ref>}
% %   \fullcite{<ref>}
% %   \footfullcite{<ref>}
%
% % ATTENTION
% % Compilation: pdflatex > biber > pdflatex > pdflatex.
%
% % Calls refs.bib at preamble with:
% \addbibresource{config/refs.bib}

% abntex2cite -------------------------------

\usepackage[
  alf,
  abnt-emphasize=bf,
  abnt-etal-list=2,
  abnt-and-type=&]{abntex2cite}
% Use:
%   \cite{<ref>}
%   \citeonline{<ref>}

% ATTENTION
% Compilation: pdflatex > bibtex > pdflatex > pdflatex.

% Calls refs.bib at last frame with:
% \bibliography{config.refs}

\let\oldbibliography\thebibliography
\renewcommand{\thebibliography}[1]{%
  \oldbibliography{#1}%
  \setlength{\itemsep}{1em}%
}

%--------------------------------------------

\usepackage{pgfgantt}

%=======================================================================
%=======================================================================
\IfFileExists{upquote.sty}{\usepackage{upquote}}{}
\begin{document}

\frame{\titlepage}

% Tabela de conteúdo no início dos slides.
 \begin{frame}{Conteúdo}
   \small{\tableofcontents}
 \end{frame}

%-----------------------------------------------------------------------




%-----------------------------------------------------------------------
\section{Modelos multivariados de covariância linear generalizada}
%-----------------------------------------------------------------------

\begin{frame}

\frametitle{Modelos multivariados de covariância linear generalizada}

\begin{itemize}
   
   \itemsep 2ex
   
  \item Configuram uma estrutura geral para análise via modelos de regressão.
  
  \item Comporta múltiplas respostas de diferentes naturezas.
  
    \item Pode-se ajustar modelos com diferentes preditores e distribuições para cada resposta.
  
  \item Os modelos levam em conta a correlação entre indivíduos do conjunto de dados. 

  \item Contornam as três mais importantes restrições das classes tradicionais de modelos de regressão:
  
    \begin{enumerate}
    
      \item A incapacidade de lidar com observações dependentes. 
      \item A incapacidade de lidar com múltiplas respostas simultaneamente.
      \item Leque reduzido de distribuições disponíveis. 

    \end{enumerate}

\end{itemize}

\end{frame}

%-----------------------------------------------------------------------

\begin{frame}
\frametitle{Modelos multivariados de covariância linear generalizada}

\begin{itemize}

\itemsep 2ex
  
  \item Parâmetros estimados nos McGLMs:
    \begin{enumerate}
      \item \textbf{Regressão}: efeito das variáveis explicativas sobre as respostas.
      \item \textbf{Dispersão}: impacto da correlação entre unidades.
      \item \textbf{Potência}: indicativo de qual distribuição se adequa ao problema.
      \item \textbf{Correlação}: relação entre respostas.
    \end{enumerate}
  
  \item Todas estas quantidades são interpretáveis e são estimadas com base nos dados.
  
  \item A estimação é feita por meio de \textbf{funções de estimação}.
    \begin{enumerate}
      \item \textbf{Função quasi-score} para parâmetros de regressão. 
      \item \textbf{Função de estimação de Pearson} para os demais parâmetros. 
      \item Para resolver o sistema de equações faz-se uso do algoritmo
\textbf{Chaser modificado}.
    \end{enumerate}

\end{itemize}

\end{frame}

%-----------------------------------------------------------------------
\section{Testes de hipóteses em modelos multivariados de covariância linear generalizada}
%-----------------------------------------------------------------------

\begin{frame}
  \frametitle{Testes de hipóteses em modelos de regressão}
  \begin{itemize}
    \itemsep 2ex
  
  \item Usados para verificar se a retirada de determinada variável explicativa do modelo geraria uma perda no ajuste.
  
  \item Os três testes mais usados são:

    \begin{itemize}
      \item O teste da razão de verossimilhanças.
      \item O teste Wald.
      \item O teste do multiplicador de lagrange, também conhecido como teste escore.
    \end{itemize}
  
  \item São baseados na função de verossimilhança dos modelos.
  
  \item São assintóticamente equivalentes.
  
  \end{itemize}
  
\end{frame}

%-----------------------------------------------------------------------

\begin{frame}
  \frametitle{ANOVA \& MANOVA}

  \begin{itemize}
    \itemsep 2ex

  \item Formas de \textbf{avaliar a significância} de cada uma das variáveis de uma forma procedural.  
  
  \item Consiste em efetuar testes sucessivos impondo \textbf{restrições ao modelo} original. 

  \item O objetivo é testar se a ausência de determinada variável gera perda ao modelo. 

      \item Os resultados são sumarizados numa tabela, o chamado \textbf{quadro de análise de variância}.

  \item Na ANOVA, avalia-se a relevância das variáveis sobre uma única resposta. 

  \item Na MANOVA, avalia-se a relevância das variáveis sobre mais de uma resposta. 
  
  \end{itemize}

\end{frame}

%-----------------------------------------------------------------------

\begin{frame}
  \frametitle{Proposta}

\begin{itemize}

  \itemsep 2ex

  \item Desenvolvimento de testes de hipóteses para avaliação dos parâmetros de McGLMs, através de uma adaptação do teste Wald.

  \item Adaptar o teste Wald para realização de testes de hipóteses gerais sobre parâmetros de McGLMs. 
  
  \item Implementar funções para efetuar tais testes, bem como funções para efetuar ANOVAs e MANOVAs para os McGLMs. 

  \item Avaliar as propriedades e comportamento dos testes propostos com base em estudos de simulação.

  \item Motivar o potencial de aplicação das metodologias discutidas com base na aplicação a conjuntos de dados reais.

  \end{enumerate}

\end{frame}

%-----------------------------------------------------------------------

\section{Adaptação do teste Wald para os McGLM}

% -----------------------------------------------------------------

\begin{frame}
\frametitle{Hipóteses}

$$H_0: \boldsymbol{L}\boldsymbol{\theta_{\beta,\tau,p}} = \boldsymbol{c} \ vs \ H_1: \boldsymbol{L}\boldsymbol{\theta_{\beta,\tau,p}} \neq \boldsymbol{c}.$$ 

Em que: 

\begin{itemize}
  
  \item Em que $\boldsymbol{L}$ é a matriz de especificação das hipóteses a serem testadas, tem dimensão $s \times h$. 
  
  \item $\boldsymbol{\theta_{\beta,\tau,p}}$ é o vetor de dimensão $h \times 1$ de parâmetros de regressão, dispersão e potência do modelo. 
  
  \item $\boldsymbol{c}$ é um vetor de dimensão $s \times 1$ com os valores sob hipótese nula.

\end{itemize}

\end{frame}

% -----------------------------------------------------------------

\begin{frame}
\frametitle{Estatística de teste}

$$W = (\boldsymbol{L\hat\theta_{\beta,\tau,p}} - \boldsymbol{c})^T \ (\boldsymbol{L \ J_{\boldsymbol{{\beta,\tau,p}}}^{-1} \ L^T})^{-1} \ (\boldsymbol{L\hat\theta_{\beta,\tau,p}} - \boldsymbol{c}).$$

Em que: 

\begin{itemize}
  \item $\boldsymbol{L}$ é a matriz da especificação das hipóteses, tem dimensão $s \times h$. 

  \item $\boldsymbol{\hat\theta_{\beta,\tau,p}}$ é o vetor de dimensão $h \times 1$ com todas as estimativas dos parâmetros de regressão, dispersão e potência. 

  \item $\boldsymbol{c}$ é um vetor de dimensão $s \times 1$ com os valores sob hipótese nula. 

  \item $J_{\boldsymbol{{\beta,\tau,p}}}^{-1}$ é a inversa da matriz de informação de Godambe desconsiderando os parâmetros de correlação, de dimensão $h \times h$. 
  
  \item $W \sim \chi^2_s$

\end{itemize}

\end{frame}

%-----------------------------------------------------------------------

\begin{frame}

\frametitle{ANOVA \& MANOVA via teste Wald}

\begin{itemize}
    \itemsep 2ex

  \item Com base na adaptação do teste Wald propostas, buscamos propor ANOVAs e MANOVAs via teste Wald.

  \item Propomos 3 tipos diferentes de análises de variância, nomeadas tipo I, II e III.

  \item Cada linha do quadro corresponde uma hipótese. Portanto, basta especificar uma matriz $\boldsymbol{L}$.

  \item Os procedimentos para análise de variância retornam um quadro para cada resposta.

  \item Os procedimentos para análise variância multivariadas retornam um único quadro. 
\end{itemize}

\end{frame}

%-----------------------------------------------------------------------
\section{Teste de comparações múltiplas}
%-----------------------------------------------------------------------

\begin{frame}

\frametitle{Teste de comparações múltiplas}

\begin{itemize}
    \itemsep 2ex

  \item Utilizados quando a análise de variância aponta como conclusão a existência de efeito significativo dos parâmetros associados a uma variável categórica.

  \item Isto é, há ao menos uma diferença significativa entre os níveis do fator.

  \item O teste de comparações múltiplas é utilizado para determinar onde estão estas diferenças. 
  
  \item Através da adaptação do teste Wald é possível chegar a procedimentos para realização de testes de comparações múltiplas.
  
\end{itemize}

\end{frame}

%-----------------------------------------------------------------------

\begin{frame}

\frametitle{Teste de comparações múltiplas}

\begin{itemize}
    \itemsep 2ex

  \item Por exemplo, suponha que há no modelo uma variável categórica $X$ de três níveis: A, B e C.

  \item Um dos níveis será definido como categoria de referência e serão estimados os parâmetros que medem a diferença da referência pras demais.

  \item A análise de variância mostrará se há efeito da variável $X$ no modelo, isto é, se os valores da resposta estão associados aos níveis de $X$.
  
  \item Contudo este resultado não nos mostrará se os valores da resposta diferem de A para B, ou de A para C, ou ainda se B difere de C.
  
  \item Este papel é dos testes de comparações múltiplas.
  
\end{itemize}

\end{frame}

%-----------------------------------------------------------------------

\begin{frame}

\frametitle{Passos}

\begin{enumerate}
    \itemsep 2ex

  \item Obter a matriz de combinações lineares dos parâmetros dos modelos que resultam nas médias ajustadas (geralmente denotada por \boldsymbol{L}).

  \item Gerar a matriz de contrastes dada pela subtração duas a duas das linhas da matriz L (geralmente denotada por \boldsymbol{K}).

  \item Selecionar as linhas da matriz \boldsymbol{K} e usar como matriz de especificação de hipóteses do teste Wald.

  \item Se for uma hipótese sobre cada uma das respostas, todas sujeitas ao mesmo preditor, basta aumentar a \boldsymbol{K} com um produto de kronecker.

  \item Se for uma hipótese por resposta, basta selecionar o vetor de estimativas e a matriz \boldsymbol{J} para a resposta específica.
  
\end{enumerate}

\end{frame}

%-----------------------------------------------------------------------
\subsection{Exemplo 1}
%-----------------------------------------------------------------------

\begin{frame}

\frametitle{Exemplo1}

\begin{itemize}
    \itemsep 2ex

  \item Variávei resposta $Y$ sujeita a uma variável explicativa $X$ de 4 níveis (A, B, C, D).

  \item $g(\mu) = \beta_0 + \beta_1[X=B] + \beta_2[X=C] + \beta_3[X=D]$.

  \item $h\left \{ \boldsymbol{\Omega}(\boldsymbol{\tau}) \right \} = \tau_{0}Z_0$.

  \item $
    \boldsymbol{L} = 
      \begin{matrix}
        A\\ 
        B\\ 
        C\\ 
        D 
      \end{matrix} 
    \begin{bmatrix}
      1 & 0 & 0 & 0\\ 
      1 & 1 & 0 & 0\\ 
      1 & 0 & 1 & 0\\ 
      1 & 0 & 0 & 1 
    \end{bmatrix}
    \hspace{1cm}
    \boldsymbol{K} = 
      \begin{matrix}
        A-B\\ 
        A-C\\ 
        A-D\\ 
        B-C\\
        B-D\\
        C-D\\ 
      \end{matrix} 
    \begin{bmatrix}
      0 & -1 &  0 &  0\\ 
      0 &  0 & -1 &  0\\ 
      0 &  0 &  0 & -1\\ 
      0 &  1 & -1 &  0\\ 
      0 &  1 &  0 & -1\\ 
      0 &  0 &  1 & -1 
    \end{bmatrix}
$
\end{itemize}

\end{frame}

%-----------------------------------------------------------------------
\subsection{Exemplo 2}
%-----------------------------------------------------------------------

\begin{frame}

\frametitle{Exemplo 2}

\begin{itemize}
    \itemsep 2ex

  \item Variávei resposta $Y$ sujeita a uma variável explicativa $X_1$ de 3 níveis (A, B, C) e $X_2$ de 2 níveis (D, E).

  \item $g(\mu) = \beta_0 + \beta_1[X_1=B] + \beta_2[X_1=C] + \beta_3[X_2=E]$.

  \item $h\left \{ \boldsymbol{\Omega}(\boldsymbol{\tau}) \right \} = \tau_{0}Z_0$.

\end{itemize}

\end{frame}

%-----------------------------------------------------------------------

\begin{frame}

\frametitle{Exemplo 2}

    \boldsymbol{L} = 
      \begin{matrix}
        A:D\\ 
        B:D\\ 
        C:D\\
        A:E\\
        B:E\\
        C:E
      \end{matrix} 
    \begin{bmatrix}
      1 & 0 & 0 & 0\\ 
      1 & 1 & 0 & 0\\ 
      1 & 0 & 1 & 0\\ 
      1 & 0 & 0 & 1\\ 
      1 & 1 & 0 & 1\\ 
      1 & 0 & 1 & 1 
    \end{bmatrix}
    \hspace{1cm}
    \boldsymbol{K} = 
      \begin{matrix}
        A:D-B:D\\ 
        A:D-C:D\\ 
        A:D-A:E\\ 
        A:D-B:E\\
        A:D-C:E\\
        B:D-C:D\\
        B:D-A:E\\
        B:D-B:E\\
        B:D-C:E\\
        C:D-A:E\\
        C:D-B:E\\
        C:D-C:E\\
        A:E-B:E\\
        A:E-C:E\\
        B:E-C:E
      \end{matrix} 
    \begin{bmatrix}
      0 & -1 &  0 &  0 \\ 
      0 &  0 & -1 &  0 \\ 
      0 &  0 &  0 & -1 \\ 
      0 & -1 &  0 & -1 \\ 
      0 &  0 & -1 & -1 \\ 
      0 &  1 & -1 &  0 \\ 
      0 &  1 &  0 & -1 \\ 
      0 &  0 &  0 & -1 \\ 
      0 &  1 & -1 & -1 \\ 
      0 &  0 &  1 & -1 \\ 
      0 & -1 &  1 & -1 \\ 
      0 &  0 &  0 & -1 \\ 
      0 & -1 &  0 & -0 \\ 
      0 &  0 & -1 & -0 \\ 
      0 &  1 & -1 & -0 
    \end{bmatrix}

\end{frame}

%-----------------------------------------------------------------------
\section{Implementações}
%-----------------------------------------------------------------------

\begin{frame}

\frametitle{Funções}

\begin{itemize}
    \itemsep 2ex

  \item \textbf{mc\_multcomp}: função que faz comparações múltiplas para cada uma das respostas.
  
  \item \textbf{mc\_mult\_multcomp}: função que faz comparações múltiplas para todas as respostas quando sujeitas ao mesmo preditor.
  
\end{itemize}

\vspace{1cm}

(Mostrar no R)

\end{frame}

%-----------------------------------------------------------------------
\section{Toy examples}
%-----------------------------------------------------------------------

\begin{frame}[c, allowframebreaks]

\begin{center}

  {\huge \href{https://lineu96.github.io/st/}{Obrigado!}}
  
  \vspace{0.5cm}
    
  {\normalsize \href{https://lineu96.github.io/st/}{Lineu Alberto Cavazani de Freitas}}
  
  {\normalsize \href{https://lineu96.github.io/st/}{lineuacf@gmail.com}}
  
  {\normalsize \href{https://lineu96.github.io/st/}{https://lineu96.github.io/st/}}
  
\end{center}

\begin{center}
  % \includegraphics[height=1.8cm]{img/capes_tp2.png}\hspace{2em}
  \includegraphics[height=1.8cm]{img/ufpr-transparent.png}\hspace{2em}
  % \includegraphics[height=1.8cm]{img/dsbd-2x2-trans.png}
\end{center}

\end{frame}

%--------------------------------------------------

\end{document}
