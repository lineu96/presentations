%---- preamble-chunk-rnw.tex -------------------------------------------

% Knitr.

% ATTENTION: this needs `\usepackage{xcolor}'.
\definecolor{color_line}{HTML}{333333}
\definecolor{color_back}{HTML}{DDDDDD}

% Tamanho de fonte e distância entre linhas.
% \renewenvironment{knitrout}{
%   \renewcommand{\baselinestretch}{0.75}%\tiny
% }{}

% Tamanho de fonte e distância entre linhas.
\renewenvironment{knitrout}{%
 \setlength{\topsep}{-0.25ex}
 \renewcommand{\baselinestretch}{0.65}
 \footnotesize
}{}

% R output e todo `verbatim'.
\makeatletter
% \def\verbatim@font{\linespread{0.9}\textit\normalfont\ttfamily\footnotesize}
\def\verbatim@font{\linespread{0.9}\ttfamily\footnotesize}
\makeatother

% Cor de fundo e margens do `verbatim'.
\let\oldv\verbatim
\let\oldendv\endverbatim

\def\verbatim{%
  \par\setbox0\vbox\bgroup % Abre grupo.
  \vspace{-5px}            % Reduz margem superior.
  \oldv                    % Chama abertura do verbatim.
}
\def\endverbatim{%
  \oldendv                 % Chama encerramento do verbatim.
  % \vspace{0cm}           % Controla margem inferior.
  \egroup\fboxsep10px      % Fecha grupo.
  \usebox0
  % \noindent{\colorbox{color_back}{\usebox0}}\par
}

%-----------------------------------------------------------------------
